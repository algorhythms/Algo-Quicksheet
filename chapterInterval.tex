\chapter{Interval}


\section{Introduction}
\rih{Two-way range.} The current scanning node as the pivot, need to scan its left neighbors and right neighbors. 
$$
|\leftarrow p \rightarrow |
$$

If the relationship between the pivot and its neighbors is symmetric, since scanning range is $[i-k, i+k]$ and iterating from left to right, only consider $[i-k, i]$ to avoid duplication.
$$
|\leftarrow p
$$

\section{Event-Driven Algorithm}
\subsection{Introduction}
The core philosophy of event-driven algorithm:
\begin{enumerate}
\item \textbf{Event}: define \textit{event}; the event are sorted by time of appearance.
\item \textbf{Heap}: define \textit{heap meaning}.
\item \textbf{Transition}: define \textit{transition functions} among events impacting the.
heap. 
\end{enumerate} 

\subsection{Questions}
\runinhead{Maximal Overlaps.} Given a list of number intervals, find max number of overlapping
intervals. 
\runinhead{Clues:}
\begin{enumerate}
\item \textbf{Event}: Every new start of an interval is an event. Scan the sorted intervals (sort the interval by \textit{start}).
\item \textbf{Heap meaning}: Heap stores the \textit{end} of the interval. 
\item \textbf{Transition}: Put the ending time into heap, and pop the ending time earlier than the new start time from heap.
\end{enumerate}

\begin{python}
def max_overlapping(intervals):
    maxa = 0
    intervals.sort(key=operator.attrgetter("start"))
    h_end = []
    for itvl in intervals:
        heapq.heappush(end_heap, itvl.end)
        
        while h_end and h_end[0] <= itvl.start:
            heapq.heappop(h_end)

        maxa = max(maxa, len(h_end))

    return maxa
\end{python}

