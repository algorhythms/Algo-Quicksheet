\chapter{Combinatorics}
\section{Basics}
\subsection{Considerations}
\begin{enumerate}
\item Does \textbf{order} matter? Does the \textbf{timing} of choice matter?
\item Are the object draws \textbf{repeatable}?
\item Are the objects partially \textbf{duplicated}?
\end{enumerate}
If order does not matter or repeated objects, you can pre-set the order. 
\subsection{Basic formula}
\begin{eqnarray*}
&& {n \choose k} = {n \choose n-k} \\
&& {n \choose k} = \frac{n!}{k!(n-k)!} \\
&& {n\choose k} = {n-1\choose k} + {n-1 \choose k-1}
\end{eqnarray*}
The 1st equation is complimentary. Considering a binary choice $x_i \in {0, 1}$ indicating whether to choose an item \pyinline{e}. Choosing $k$ objects (assigning $1$) directly means not choosing $n-k$ objects (assigning $0$). 
\begin{eqnarray*}
X &&= \{x_0, x_1, ... x_{n-1}\} \\
x_i &&\in {0, 1} \\
\sum_i{x_i} &&= k \\
\sum_i{1 - x_i} &&= n - k \\
|X| &&= n
\end{eqnarray*}


The 2nd equation is deduping. Considering a set of objects $A = \{a, b, c, d, e\}$. $n = 5$. The number of total permutations is $5!$. If we want to pick $k=3$ objects: $list = [\alpha_0, \alpha_1, \alpha_2]$, but the order does not matter or element are repeatble, it becomes $set = \{\alpha_0, \alpha_1, \alpha_2\}$. It means $3!$ permutation $list$ is degraded to a single $set$. At the same time, the complimentary $list_{comp} = [\alpha_3, \alpha_4]$ is degraded to a single $set_{comp}$. For example, we are choosing $\{a, b, c\}$ and when the program is processing the permutation $abcde$ or $adbec$, we need to count duplicates for deduping. 
\begin{eqnarray*}
\mathbf{abc}de &&= \mathbf{bac}de \\
\mathbf{abc}de && = \mathbf{abc}ed \\
\mathbf{a}d\mathbf{b}e\mathbf{c} &&= \mathbf{b}d\mathbf{a}e\mathbf{c} \\
\mathbf{a}d\mathbf{b}e\mathbf{c} && = \mathbf{a}e\mathbf{b}d\mathbf{c} \\
ret &&= \frac{5!}{3! \cdot 2!}
\end{eqnarray*}

The 3rd equation is DP. Let $F_{n, k}$ be the number of combinations of choosing $k$ elements from $n$ elements. We can either pick the $n$-th item or not. 
$$
F_{n,k} = F_{n-1, k} + F_{n-1, k-1}
$$

\subsection{N objects, K ceils. Stars \& Bars}
When $N=10, K=3$:
$$
x_1 + x_2 + x_3 = 10
$$
is equivalent to
$$
*****|**|***
$$

, notice that $*$ are non-order, and it is possible to have 
$$
*****||*****
$$
\\
then the formula is:
$$
{n+r \choose r}
$$

,where $r=k-1$. 
\\
\rih{Intuitively}, the meaning is to choose $r$ objects from $n+r$ objects to become the $|$.

\runinhead{Unique paths.} Given a $m \times n$ matrix, starting from $(0, 0)$, ending at $(m-1, n-1)$, can only goes down or right. What is the number of unique paths?

Let $F_{i, j}$ be the number of unique paths at \pyinline{[i][j]}. 
$$
F_{i, j} = F_{i-1, j} + F_{i, j-1}
$$

\subsection{N objects, K types} \label{N_objects_K_types}
What is the number of permutation of $N$ objects with $K$ different types:
\begin{align*}
ret &= \frac{A_N^N}{\prod_{k=1}^K{A_{sz(k)}^{sz(k)}}} \\
&= \frac{N!}{\prod_{k} sz[k]!}
\end{align*}

\subsection{Inclusion–Exclusion Principle}
\begin{figure}[hbtp]
\centering
\subfloat{\includegraphics[width=0.8\linewidth]{500px-Inclusion-exclusion}}
\caption{Inclusion–exclusion principle}
\label{fig:500px-Inclusion-exclusion}
\end{figure}
\begin{eqnarray*}
|A \cup B \cup C| = |A| + |B| + |C| \\ - |A \cap B| - |A \cap C| - |B \cap C| \\ + |A \cap B \cap C|
\end{eqnarray*}
Generally,
$$
\Biggl|\bigcup_{i=1}^n A_i\Biggr| = \sum_{k = 1}^{n} (-1)^{k+1} \left( \sum_{1 \leq i_{1} < \cdots < i_{k} \leq n} \left| A_{i_{1}} \cap \cdots \cap A_{i_{k}} \right| \right)
$$
\section{Combinations with Limited Repetitions}
Determine the number of combinations of 10 letters (order does not matter) that can be formed from 12 letters of 3A, 4B, 5C, i.e. multisets $S=\{3.A, 4.B, 5.C\}$. 

\subsection{Basic Solution}
$|S| = 12$. If there are unlimited the number of any of the letter, it is ${10+2 \choose 10}$ by stars and bars of $x1+x2+x3=10$; then we get the universal set, 
$$
|U|={10+2 \choose 10}={10+2 \choose 2}
$$

Let $P_A$ be the set that a 10-combination has \rih{more than} 3A. $P_B$...4B. $P_C$...5C. 

The result is:
\begin{align*}
|3A \cap 4B \cap 5C| = & |U|\\
& - \sum{(|P_i|\cdot \forall i)} \\
& + \sum{(|P_i \cap P_j|\cdot \forall i,j)}\\
& - \sum{(|P_i \cap P_j \cap P_k|\cdot \forall i,j,k)}
\end{align*}

To calculate $|P_i|$, take $|P_A|$ as an example. $P_A$ means at least 4A -- if we take any one of these 10-combinations in $P_A$ and remove 4A we are left with a 6-combination with unlimited on the numbers of letters, including $A$; thus,
$$
|P_A|={6+2 \choose 2}
$$

Similarly, we can get $P_B, P_C$.

To calculate $|P_i \cap P_j|$, take $|P_A \cap P_B|$ as an example for 4A and 5B; thus,
$$
|P_A \cap P_B| = {1+2 \choose 2}
$$

Similarly, we can get other $|P_i \cap P_j|$.

Similarly, we can get other $|P_i \cap P_j \cap P_k|$.
\subsection{Algebra Interpretation}
The number of 10-combinations that can be made from 3A, 4B, 5C is found from the coefficient of $x^{10}$ in the expansion of:
$$
(1+x+x^2+x^3)(1+x+x^2+x^3+x^4)(1+x+x^2+x^3+x^4+x^5)
$$

And we know:
\begin{eqnarray*}
1+x+x^2+x^3         = (1-x^4)/(1-x)  \\
1+x+x^2+x^3+x^4     = (1-x^5)/(1-x)  \\
1+x+x^2+x^3+x^4+x^5 = (1-x^6)/(1-x)  \\
\end{eqnarray*}


We expand the formula, although the  naive way of getting the coefficient of $x^{10}$ is tedious. 

\section{Permutation}

\subsection{$k$-th permutation}
Given $n$ and $k$, return the $k$-th permutation sequence. $k\in [1, n!]$. $O(n!)$ in time complexity is easy.
\begin{python}
def getPermutation(n, k):
    A = [i + 1 for i in range(n)]
    ret = []
    genPermutation(A, 0, [], ret)
    ret.sort()  # requried 
    return ret[k - 1]

def genPermutation(A, idx, cur, ret):
    if idx == len(A):
        ret.append("".join(map(str, cur)))

    for j in range(idx, len(A)):
        A[idx], A[j] = A[j], A[idx]
        cur.append(A[idx])
        genPermutation(A, idx + 1, cur, ret)
        A[j], A[idx] = A[idx], A[j]
        cur.pop()
\end{python}

Can we do it in $O(nk)$ or less?

Reversed Cantor Expansion

Core clues:
\begin{enumerate}
\item \pyinline{A = [1, 2, ..., n]}

Suppose for $n$ element, the $k$-th permutation is:

\pyinline{ret = [a0, a1, a2, ..., an-1]}. $A_i$ is different from $a_i$.
\item \rih{Basic case.} Since \pyinline{[a1, a3, ..., an-1]} has $(n-1)!$ permutations,
if $k < (n-1)!, a_0 = A_0$ (first element in array), else $a_0 = A_{k/(n-1)!}$

\item Recursively, (or iteratively), calculate the values at each position. Similar to Radix. 
\begin{enumerate}
\item $a_0 = A_{k_0/(n-1)!}$, where $k_0 = k$
\item $a_1 = A_{k_1/(n-2)!}$, where $k_1 = k_0\%(n-1)!$ in the remaining array $A$
\item $a_2 = A_{k_2/(n-3)!}$, where $k_2 = k_1\%(n-2)!$ in the remaining array $A$
\end{enumerate}
\end{enumerate}
\begin{python}
def getPermutation(self, n, k):
    k -= 1  # since k was 1-indexed

    A = [i + 1 for i in range(n)]
    # k %= math.factorial(n)  # if k > n!
    ret = []
    for i in range(n-1, -1, -1):
        idx = k // math.factorial(i) 
        cur = A.pop(idx)
        ret.append(cur)
        k = k % math.factorial(i)

    return "".join(map(str, ret))
\end{python}

\subsection{Numbers counting}
\runinhead{Count numbers with unique digit.} Given a non-negative integer n, count all numbers with unique digits, $x$, where $0 \leq x < 10^n$.

Digit by digit: 
\begin{enumerate}
\item The 1st digit has 10 possibilities. The 2nd digit has 9 possibilities. Therefore it seems to be $A_{10}^n$.
\item Exception: The first digit cannot be 0. Therefore it is $9\times 9\times 8\times ...\times (10-i)$
\end{enumerate}



\section{Catalan Number}\label{section:catalanNumber}
\subsection{Math}
\runinhead{Definition.}
$$
C_n = {2n\choose n} - {2n\choose n+1} = {1\over n+1}{2n\choose n} \quad\text{ for }n\ge 0
$$
\runinhead{Proof.} Proof of Calatan Number $C_n ={2n\choose n} - {2n\choose n+1}$. Objective: count the number of paths in $n\times n$ grid without exceeding the main diagonal. 
\begin{itemize}
\begin{figure}[]
    \centerline{\includegraphics[height = 1.6in]{catalan_proof}}
    \caption{Monotonic Paths}
  \label{fig:catalanProof}
\end{figure}
\item Total monotonic paths including both exceeding and not exceeding - from $2n$ choose $n$ right and $n$ up:
$$
{2n\choose n}
$$
\item Find out how many combinations that exceeds the main diagonal. Flip all the exceeding lines at the line just above the diagonal line, we will see a rectangle of size $(n-1) * (n+1)$ has formed. Thus the total number of monotoic paths in the new rectange is - from $2n$ choose $n-1$ right and $n+1$ up:
$$
{n-1+n+1\choose n-1}
$$
\item Thus, the number of path without \textit{exceeding} (i.e. passing the diagonal line) is: 
\begin{align*}
C_n &= {2n\choose n} - {2n\choose n-1}\\ 
&={2n\choose n} - {2n\choose n+1}
\end{align*}
\end{itemize}

\subsection{Applications}
The paths in Figure \ref{fig:catalanProof} can be abstracted to anything that at any time \#right $\geq$ \#up. 
\runinhead{\#Parentheses.}Number of different ways of adding parentheses. At any time, \#\pyinline{(} $\geq$ \#\pyinline{)}.
\runinhead{\#Full binary trees.}Number of different full bindary tree. A full binary tree is a tree in which every node has either 0 or 2 children. Consider it as a set of same binary operators with their operands. Reduce this problem to \#Parentheses. 
\begin{figure}[hbtp]
\centering
\subfloat{\includegraphics[width=\linewidth]{Catalan_number_binary_tree_example}}
\caption{\#Full bindary trees. Circles are operators; crescents are operands.}
\label{fig:NumberOfBSTs}
\end{figure}

\section{Stirling Number}
A Stirling number of the second kind (or Stirling partition number) is the number of ways to partition a set of n objects into k non-empty subsets and is denoted by $S(n,k)$ or  $\lbrace{n\atop k}\rbrace$.

$$
\left\{ {n \atop k}\right\} = \frac{1}{k!}\sum_{j=0}^{k} (-1)^{k-j} \binom{k}{j} j^n.
$$
