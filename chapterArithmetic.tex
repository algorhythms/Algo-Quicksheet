\chapter{Arithmetic}
\section{DFS}
\rih{Insert operators.} Given a string that contains only digits 0-9 and a target value, return all possibilities to add binary operators (not unary) +, -, or * between the digits so they evaluate to the target value.

Example: 
\begin{align*}
"123", 6 \rightarrow ["1+2+3", "1*2*3"] \\ 
"232", 8 \rightarrow ["2*3+2", "2+3*2"] \\
\end{align*}
Clues:
\begin{enumerate}
\item DFS
\item Special handling for multiplication - caching 
\item Detect invalid number with leading 0's
\end{enumerate}}
Code: 
\begin{python}
def addOperators(self, num, target):
  ret = []
  self.dfs(num, target, 0, "", 0, 0, ret)
  return ret

def dfs(self, num, target, pos, 
        cur_str, cur_val, 
        mul, ret
    ):
  if pos >= len(num):
    if cur_val == target:
      ret.append(cur_str)
  else:
    for i in xrange(pos, len(num)):
      if i != pos and num[pos] == '0':
        continue
      nxt_val = int(num[pos:i+1])

      if not cur_str:
        self.dfs(num, target, i+1, 
            "%d"%nxt_val, nxt_val,
            nxt_val, ret)
      else:
        self.dfs(num, target, i+1, 
            cur_str+"+%d"%nxt_val, cur_val+nxt_val, 
            nxt_val, ret)
        self.dfs(num, target, i+1, 
            cur_str+"-%d"%nxt_val, cur_val-nxt_val, 
            -nxt_val, ret)
        self.dfs(num, target, i+1, 
            cur_str+"*%d"%nxt_val, cur_val-mul+mul*nxt_val, 
            mul*nxt_val, ret)
\end{python}
\rih{Insert parenthesis.} Given a string of numbers and operators, return all possible results from computing all the different possible ways to
group numbers and operators. The valid operators are +, - and *.

Examples:
\begin{align*}
(2*(3-(4*5))) &= -34 \\
((2*3)-(4*5)) &= -14 \\
((2*(3-4))*5) &= -10 \\
(2*((3-4)*5)) &= -10 \\
(((2*3)-4)*5) &= 10
\end{align*}
Clues: Iterate the operators, divide and conquer - left parts and right parts and then combine result. \\
Code:
\begin{python}
def dfs_eval(self, nums, ops):
    ret = []
    if not ops:
        assert len(nums) == 1
        return nums

    for i, op in enumerate(ops):
        left_vals = self.dfs_eval(nums[:i+1], ops[:i])
        right_vals = self.dfs_eval(nums[i+1:], ops[i+1:])
        for l in left_vals:
            for r in right_vals:
                ret.append(self._eval(l, r, op))

    return ret
\end{python}

\section{Big Number}
\rih{Plus One.} Given a non-negative number represented as an array of digits, plus one to the number.
\begin{python}
def plusOne(self, digits):
    for i in xrange(len(digits)-1, -1, -1):
        digits[i] += 1
        if digits[i] < 10:
            return digits
        else:
            digits[i] -= 10

    # if not return within the loop 
    digits.insert(0, 1)
    return digits
\end{python}

\section{Polish Notation}
Polish Notation is in-fix while Reverse Polish Notation is post-fix. 
\subsection{Evaluate Post-fix Expressions}
Straightforward: Use a stack to store the number. Iterate the input, push stack when
hit numbers, pop stack when hit operators.
\subsection{Convert In-fix to Post-fix}
TODO

