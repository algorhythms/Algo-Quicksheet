\chapter{Balanced Search Tree}
\section{2-3 Search Tree}
\subsection{Insertion}
Insertion into a 3-node at bottom:
\begin{enumerate}
\item Add new key to the 3-node to create a temporary 4-node.
\item Move middle key of the 4-node into the parent (including root's parent).
\item Split the modified 4-node.
\item Repeat recursively up the trees as necessary.
\end{enumerate}
\begin{figure}[hbtp]
\centering
\subfloat{\includegraphics[scale=1.]{23insert1}}
\caption{Insertion 1}
\label{fig:LABEL}
\end{figure}

\begin{figure}[hbtp]
\centering
\subfloat{\includegraphics[scale=1.]{23insert2}}
\caption{insert 2}
\label{fig:LABEL}
\end{figure}

\subsection{Splitting}
Summary of splitting the tree. 
\begin{figure}[hbtp]
\centering
\subfloat{\includegraphics[scale=.60]{23splitting}}
\caption{Splitting temporary 4-ndoe summary}
\label{fig:splitting}
\end{figure}

\subsection{Properties}
When inserting a new key into a 2-3 tree, under which one of the following scenarios must the height of the 2-3 tree increase by one? When every node on the search path from the root is a 3-node

\section{Red-Black Tree}\label{rbtree}
\subsection{Properties}
Red-black tree is an implementation of 2-3 tree using \textbf{leaning-left red link}. \begin{figure}[hbtp]
\centering
\subfloat{\includegraphics[scale=1.1]{rbtree11}}
\caption{RB-tree and 2-3 tree}
\label{fig:LABEL}
\end{figure}
The height of the RB-tree is at most $2\lg N$ where alternating red and black links. Red is the special link while black is the default link. 

\runinhead{Perfect black balance.}Every path from root to null link has the same number of black links.
\subsection{Operations}
\runinhead{Elementary operations:}
\begin{enumerate}
\item Left rotation: orient a (temporarily) right-leaning red link to lean left. Rotate leftward. 
\item Right rotation: orient a (temporarily) left-leaning red link to lean right. 
\item Color flip: Recolor to split a (temporary) 4-node. Rotate rightward. 
\end{enumerate}
\begin{figure}[hbtp]
\centering
\subfloat{\includegraphics[scale=1.20]{rbrotate}}
\caption{Rotate left/right}
\label{fig:LABEL}
\end{figure}

\begin{figure}[hbtp]
\centering
\subfloat{\includegraphics[scale=1.20]{rbflip}}
\caption{Flip colors}
\label{fig:LABEL}
\end{figure}

\runinhead{Insertion.} When doing insertion, from the child's perspective, need to have the information of current leaning direction and parent's color. Or from the parent's perspective - need to have the information of children's and grandchildren's color and directions.

For every new insertion, the node is always attached with red links. 

The following code is the simplest version of RB-tree insertion: 

\begin{java}
Node put(Node h, Key key, Value val) {
  if (h == null)  // std red insert (link to parent).
    return new Node(key, val, 1, RED);
  int cmp = key.compareTo(h.key);
  if      (cmp < 0) h.left  = put(h.left,  key, val);
  else if (cmp > 0) h.right = put(h.right, key, val);
  else h.val = val; // pass

  if (isRed(h.right) && !isRed(h.left))    
    h = rotateLeft(h);
  if (isRed(h.left) && isRed(h.left.left)) 
    h = rotateRight(h);
  if (isRed(h.left) && isRed(h.right))     
    flipColors(h);

  h.N = 1+size(h.left)+size(h.right);
  return h; 
}
\end{java}

Rotate left, rotate right, then flip colors.

\runinhead{Illustration of cases.} Insert into a single 2-node: Figure-\ref{fig:rb_2}. Insert into a single 3-node: Figure-\ref{fig:rb_3}
\begin{figure}[t]
\begin{tabular}{cc}
  \includegraphics[height = 1.7in]{rb_left} &
  \includegraphics[height = 1.7in]{rb_right}\\
\end{tabular}
\caption{(a) smaller than 2-node (b) larger than 2-nod}
\label{fig:rb_2}
\end{figure}

\begin{figure}[t]
        \centerline{\includegraphics[height = 2.8in]{rb_3_left_right_btw}}
        \caption{(a) larger than 3-node (b) smaller than 3-node (c) between 3-node.}
    \label{fig:rb_3}
\end{figure}

\runinhead{Deletion.} Deletion is more complicated. 

\section{B-Tree}
B-tree is the generalization of 2-3 tree. 
\begin{figure}[hbtp]
\centering
\subfloat{\includegraphics[scale=.50]{b-tree}}
\caption{B-Tree}
\label{fig:b-tree}
\end{figure}
\subsection{Bascis}
Half-full principle: 

\begin{tabular}{lll}
\hline\noalign{\smallskip}
\textbf{Attrs} & \textbf{Non-leaf} & \textbf{Leaf} \\
\noalign{\smallskip}\hline\noalign{\smallskip}
Ptrs & \lceil\frac{n+1}{2}\rceil & \lfloor\frac{n+1}{2}\rfloor \\
\noalign{\smallskip}\hline\noalign{
\caption{Nodes at least half-full}
\end{tabular}

\subsection{Operations}
Core clues
\begin{enumerate}
\item \textbf{Split \& Up}: split half, move up the RIGHT node's FIRST of split nodes
recursively
\item \textbf{Remove}: the node moved up should be removed in the original node UNLESS it is a leaf
node. 
\end{enumerate}

\section{AVL Tree}
TODO
