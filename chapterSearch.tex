\chapter{Search}

\section{Binary Search}
\runinhead{Variants:}
\begin{enumerate}
\item get the idx equal or just lower (floor)
\item get the idx equal or just higher (ceil)
\item \pyinline{bisect_left}
\item \pyinline{bisect_right} $\equiv$ \pyinline{bisect} 
\end{enumerate}

Note the subtle differences.

\subsection{idx equal or just lower}
Binary search, get the idx of the element equal to or just lower than the target. The returned idx is the $A_{idx} \leq target$. It is possible to return $-1$. It is different from the \pyinline{bisect_lect}.

\runinhead{Core clues:}
\begin{enumerate}
\item To get ``equal'', \pyinline{return mid}.
\item To get ``just lower'', \pyinline{return lo-1}.
\end{enumerate}
$A_{idx} \leq target$.
\begin{python}
def bin_search(self, A, t, lo=0, hi=None):
    if hi is None: hi = len(A)
    
    while lo < hi:
        mid = (lo+hi) // 2
        if A[mid] == t:  return mid
        elif A[mid] < t: lo = mid+1
        else:            hi = mid

    return lo-1
\end{python}

Using \pyinline{bisect_left} with multiple pre-checks to simply the find process.
\begin{python}
def find(A, v):
  # A is sorted 
  if not A:
    return None
  if v >= A[-1]:
    return A[-1]
  if v < A[0]:
    return None
    
  idx = bisect_left(A, v)
  if A[idx] == v:
    return v
  idx -= 1  # already checked before
  return A[idx]
\end{python}

\subsection{idx equal or just higher}
$A_{idx} \geq target$.
\begin{python}
def bi_search(self, A, t, lo=0, hi=None):
    if hi is None: hi = len(A)
   
    while lo < hi:
        mid = (lo+hi)/2
        if A[mid] == t:  return mid
        elif A[mid] < t: lo = mid+1
        else:            hi = mid
        
    return lo
\end{python}
\subsection{bisect\_left}
Return the index where to insert item x in list A. So if t already appears in the list,
A.insert(t) will insert just before the \textit{leftmost} t already there.

By insertion point \pyinline{i}, it means \pyinline{all(val <= x for val in A[lo:i])} for the left side and \pyinline{all(val > x for val in A[i:hi])} for the right side. \pyinline{A[i]} is the first element larger than x. 
\runinhead{Core clues:}
\begin{enumerate}
\item Move \pyinline{lo} if $A_{mid} < t$
\item Move \pyinline{hi} if $A_{mid} \geq t$
\end{enumerate}

\begin{python}
def bisect_left(A, t, lo=0, hi=None):
    if hi is None: hi = len(A)

    while lo < hi:
        mid = (lo+hi)/2
        if A[mid] < t: lo = mid+1   
        else:          hi = mid

    return lo
\end{python}

\subsection{bisect\_right}
Return the index where to insert item x in list A. So if t already appears in the list, A.insert(t) will insert just after the \textit{rightmost} x already there.
\runinhead{Core clues:}
\begin{enumerate}
\item Move \pyinline{lo} if $A_{mid} \leq t$
\item Move \pyinline{hi} if $A_{mid} > t$
\end{enumerate}
\begin{python}
def bisect_right(A, t, lo=0, hi=None):
    if hi is None: hi = len(A)

    while lo < hi:
        mid = (lo+hi)/2
        if A[mid] <= t: lo = mid+1
        else:           hi = mid 

    return lo
\end{python}
\section{Applications}
\subsection{Rotation}
\runinhead{Find Minimum in Rotated Sorted Array.} Case by case analysis. Three cases to consider:
\begin{enumerate}
\item Monotonous 
\item Trough 
\item Peak
\end{enumerate}

If the elements can be duplicated, need to detect and skip. 
\begin{python}
def find_min(self, A):
    lo = 0
    hi = len(A)
    mini = sys.maxsize
    while lo < hi:
        mid = (lo+hi)/2
        mini = min(mini, A[mid])
        if A[lo] == A[mid]:  # JUMP
            lo += 1
        elif A[lo] < A[mid] <= A[hi-1]:
            return min(mini, A[lo])
        elif A[lo] > A[mid] <= A[hi-1]:  # trough
            hi = mid
        else:  # peak
            lo = mid+1

    return mini
\end{python}
\section{Combinations}
\subsection{Extreme-value problems}\label{extremeValueProblem}
\runinhead{Longest increasing subsequence.} Array $A$.

Clues:
\begin{enumerate}
\item \pyinline{MIN}: \textit{min} of index \textit{last} value of LIS of a particular \textit{\textbf{len}}.
\item \pyinline{PI}: result table, store the $\pi$'s idx (predecessor); (optional, to build the LIS, no need if only needs to return the length of LIS)
\item \pyinline{bin_search}: For each currently scanning index \pyinline{i}, if it smaller (i.e. $\neg$ increasing), to maintain the \pyinline{MIN}, binary search to find the position to update the min value. The \pyinline{bin_search} need to find the element $\geq$ to \pyinline{A[i]}.
\end{enumerate}
\newpage
\begin{python}
def LIS(self, A):
    n = len(A)
    MIN = [-1 for _ in range(n+1)]
    k = 1
    MIN[k] = A[0]  # store value
    for v in A[1:]:
        idx = bisect.bisect_left(MIN, v, 1, k+1)
        MIN[idx] = v
        k += 1 if idx == k+1 else 0

    return k
\end{python}
If need to return the LIS itself. 
\begin{python}
    n = len(A)
    MIN = [-1 for _ in range(n+1)]
    RET = [-1 for _ in range(n)]
    l = 1
    MIN[l] = 0  # store index
    for i in range(1, n):
        if A[i] > A[MIN[l]]:
            l += 1
            MIN[l] = i

            PI[i] = MIN[l-1]  # (PI)
        else:
            j = self.bin_search(MIN, A, A[i], 1, l+1)
            MIN[j] = i

            PI[i] = MIN[j-1] if j-1 >= 1 else -1  # (PI)

    # build the LIS (RET)
    cur = MIN[l]
    ret = []
    while True:
        ret.append(A[cur])
        if PI[cur] == -1: break
        cur = PI[cur]

    ret = ret[::-1]
    print ret
\end{python}

\section{High dimensional search}
\subsection{2D}
\runinhead{2D search matrix I.} $m\times n$ mat. Integers in each row are sorted from left to right. The first integer of each row is greater than the last integer of the previous row.
$$
\begin{bmatrix}
1 & 3 & 5 & 7 \\
10 & 11 & 16 & 20 \\
23 & 30 & 34 & 50 \\
\end{bmatrix}
$$

Row column search: starting at top right corner: $O(m+n)$.

Binary search: search rows and then search columns: $O(\log m + \log n)$.


\runinhead{2D search matrix II.} $m\times n$ mat. Integers in each row are sorted from
left to right. Integers in each column are sorted in ascending from top to bottom.
$$
\begin{bmatrix}
1&   4&  7& 11& 15 \\
2&   5&  8& 12& 19 \\
3&   6&  9& 16& 22 \\
10& 13& 14& 17& 24 \\
18& 21& 23& 26& 30 \\
\end{bmatrix}
$$ 

Row column search: starting at top right corner: $O(m+n)$.

Binary search: search rows and then search columns, but upper bound row and lower bound row: 

$$O\big(\min(n\log m, m\log n)\big)$$
