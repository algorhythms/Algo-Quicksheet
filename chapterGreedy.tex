\chapter{Greedy}

\section{Introduction}
Philosophy: choose the best options at the current state without backtracking/reverting the choice in the future. 

A greedy algorithm is an algorithm that follows the problem solving heuristic of making the \textbf{local optimal} choice at each stage with the hope of finding a global optimum.

Greedy algorithm is a degraded DP since the the past substructure is not remembered.

\subsection{Proof}
The proof technique for the correctness of the greedy method. 

Proof by contradiction, the solution of greedy algorithm is $\mat{G}$ and the optimal solution is $\mat{O}$, $\mat{O}\neq \mat{G}$ (or relaxed to $|\mat{O}|\neq |\mat{G}|$). 

Two general technique it is impossible to have $\mat{O}\neq \mat{G}$:
\begin{enumerate}
\item \rih{Exchange method}: Greedy-Choice Property. Any optimal solution can be transformed into the solution produced by the greedy algorithm without worsening its quality.  
\item \rih{Stays-ahead method}: Optimal Substructure. the solution constructed by the greedy algorithm is consistently as good as or better than any other solution at every step or position
\end{enumerate}

\section{Extreme First}
\runinhead{Rearranging String $k$ distance apart.} Given a non-empty string $s$ and an integer $k$, rearrange the string such that the same characters are at least distance
$k$ from each other.

\textbf{Core clues.} 
\begin{enumerate}
\item The char with the most count put to the result first - greedy.
\item Fill every $k$ slots as cycle - greedily fill high-count char as many as possible.
\end{enumerate}

\textbf{Implementations.}
\begin{enumerate}
\item Use a heap as a way to get the char of the most count. 
\item \pyinline{while} loop till exhaust the heap
\end{enumerate}

\begin{python}
def rearrangeString(self, s, k):
  if not s or k == 0: return s

  d = defaultdict(int)
  for c in s:
    d[c] += 1

  h = []
  for char, cnt in d.items():
    heapq.heappush(h, Val(cnt, char))

  ret = []
  while h:
    cur = []
    for _ in range(k):
      if not h: 
        return "".join(ret) if len(ret) == len(s) else ""

      e = heapq.heappop(h)
      ret.append(e.val)
      e.cnt -= 1
      if e.cnt > 0:
        cur.append(e)

    for e in cur:
      heapq.heappush(h, e)

  return "".join(ret)

\end{python}

