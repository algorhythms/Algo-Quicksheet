\chapter{Graph}

\section{Basic}
\runinhead{Graph representation.} $V$ for a vertex set with a map, mapping from vertex to its neighbors. The mapping relationship represents the edges $E$.
\begin{python}
V = defaultdict(list)
\end{python}

\runinhead{Complexity.} Basic complexities:

\begin{table}

\begin{tabular}{lll}
\hline\noalign{\smallskip}
\textbf{Algorithm} & \textbf{Time}  & \textbf{Space}\\
\noalign{\smallskip}\hline\noalign{\smallskip}
dfs & $O(|E|)$ & $O(|V|), O(\text{longest path})$ \\
bfs & $O(|E|)$ & $O(|V|)$ \\
\noalign{\smallskip}\hline\noalign{\smallskip}
\end{tabular}

\caption{Time complexities}
\end{table}

\runinhead{Graph \& Tree.} For a undirected graph to be a tree, it needs to satisfied two conditions:
\begin{enumerate}
\item Acyclic
\item All connected
\end{enumerate}
\section{DFS}
\rih{Number of Islands.} The most fundamental and classical problem.
\begin{python}
11000
11000
00100
00011
Answer: 3
\end{python}
\rih{Clue}:
\begin{enumerate}
\item Iterative DFS
\end{enumerate}
\begin{python}
class Solution:
  def __init__(self):
    self.dirs = [(-1, 0), (1, 0), (0, -1), (0, 1)]

  def numIslands(self, grid):
    cnt = 0
    m = len(grid)
    n = len(grid[0])
    visited = [
      [False for _ in range(n)]
      for _ in range(m)
    ]
    for i in range(m):
      for j in range(n):
        if not visited[i][j] and grid[i][j] == "1":
          self.dfs(grid, i, j, visited)
          cnt += 1

    return cnt
\end{python}

\begin{python}
  def dfs(self, grid, i, j, visited):
    m = len(grid)
    n = len(grid[0])
    visited[i][j] = True

    for dir in self.dirs:
      I = i+dir[0]
      J = j+dir[1]
      if 0 <= I < m and 0 <= J < n
        and not visited[I][J]
        and grid[I][J] == "1":
        self.dfs(grid, I, J, visited)
\end{python}
If the islands are constantly updating and the query for number of islands is called multiple times, need to use Union-Find (Section \ref{section:unionFind}) to reduce each query's complexity from $O(mn)$ to $O(\log mn)$.

Without updating, DFS itself is enough.

\section{BFS}
\subsection{BFS with Abstract Level}
BFS goes through the vertices level by level in a queue. 

Distance can be arbitrarily defined. BFS can start with a set of vertices in abstract level of distance, not necessarily neighboring vertices.

Example: $-1$ denotes obstacles, $0$ denotes targets, calculate all other vertices' Manhattan distance to its nearest target:
$$
\begin{bmatrix}
\infty & -1 & 0 & \infty \\
\infty & \infty & \infty & -1 \\
\infty & -1 & \infty & -1 \\
0 & -1 & \infty & \infty \\
\end{bmatrix}
$$

Then it is calculated as:
$$
\begin{bmatrix}
3 & -1 & 0 & 1 \\
2 & 2 & 1 & -1 \\
1 & -1 & 2 & -1 \\
0 & -1 & 3 & 4 \\
\end{bmatrix}
$$

\rih{Code:}
\begin{python}
self.dirs = [(-1, 0), (1, 0), (0, -1), (0, 1)]

def wallsAndGates(self, mat):
  q = [
    (i, j) 
    for i, row in enumerate(mat)
    for j, cell in enumerate(row) 
    if cell == 0
  ]
  for i, j in q:
    for d in self.dirs:
      I, J = i+d[0], j+d[1]
      if 0 <= I < m and 0 <= J < n 
        and mat[I][J] > mat[i][j]+1:  # test distance
        mat[I][J] = mat[i][j]+1
        q.append((I, J))
\end{python}


\section{Detect Acyclic}
\begin{enumerate}
\item \pythoninline{path} represent the current path, and it is reset/pop after a dfs.
\item \pythoninline{visited} should be updated only in the end of the dfs.
\item For directed graph:
\begin{enumerate}
\item Should dfs for all neighbors except for vertices in \pythoninline{visited}, to avoid revisiting. For example, avoid revisiting A, B when start from C in the graph, and A, B have already been visited $C \rightarrow A \rightarrow B$.
\item Excluding predecessor \pythoninline{pi} is wrong considering the case of $A \leftrightarrow B$
\end{enumerate}
\item For undirected graph:
\begin{enumerate}
\item Should dfs for all neighbors except for the predecessor \pythoninline{pi}. $A-B$.
\item Excluding neighbors in \pythoninline{visited} is redundant, due to \pyinline{pi}. But it is okay to double check \pythoninline{visited}.
\end{enumerate}
\end{enumerate}

\subsection{Directed Graph}
Detect cycles (any) in directed graph.

\begin{python}
def dfs(self, V, v, visited, path) -> bool:
  if v in path:
    return False

  path.add(v)
  for nbr in V[v]:
    if nbr not in visited:
      if not self.dfs(V, nbr, visited, path):
        return False

  path.remove(v)
  visited.add(v)
  return True
\end{python}


\subsection{Undirected Graph}
Detect cycles (any) in undirected graph.

\begin{python}
def dfs(self, V, v, pi, visited, path):
  if v in path:
    return False

  path.add(v)
  for nbr in V[v]:
    if nbr != pi:  # nbr not in visited
      if not self.dfs(V, nbr, v, visited, path):
        return False

  path.remove(v)
  visited.add(v)
  return True
\end{python}
\section{Directed Graph}
Use \pyinline{G  = defaultdict(dict)} to represent directed graph, so that later on the edge weight can be accessed as \pyinline{G[s][e]}. The \pyinline{s} start and \pyinline{e} end are the vertices, and \pyinline{G[s][e]} returns edge weight.

\runinhead{DFS.} DFS in directed graph:
\begin{python}
def dfs(self, G, s, e, path):
    if s not in G or e not in G:
        return INVALID
    if e in G[s]:
        return G[s][e]
    for nbr in G[s]:
        if nbr not in path:
            path.add(nbr)
            val = self.dfs(G, nbr, e, path)
            if val != -1.0:
                return val * G[s][nbr]
            path.remove(nbr)

    return INVALID
\end{python}

\section{Paths}
\subsection{Euler Path - Every Edge Once} 
An Eulerian path is a path in a graph which visits every edge exactly once ($\forall e \in E$). Vertices can be repeated.

Hierholzer's algorithm to find an Euler path in a graph. The graph must be directed graph.

\runinhead{Core clue.}
\begin{enumerate}
\item The algorithm exhaustively visit all the edges during the dfs. 
\item We must \textbf{remove} the edge after visting, otherwise circle.
\item After dfs a vertex $v$'s all neighbors and path, \pyinline{appendleft} $v$. 
\end{enumerate}
\begin{python}
def findItinerary(self, tickets):
    G = defaultdict(list)
    for s, e in tickets:
        heapq.heappush(G[s], e)  # heap lexical order

    ret = deque()
    self.dfs(G, 'JFK', ret)
    return list(ret)

def dfs(self, G, cur, ret):
    while G[cur]:
        # need to remove the edge after visting
        nbr = heapq.heappop(G[cur])
        self.dfs(G, nbr, ret)

    ret.appendleft(cur)
\end{python}

Alternatively, instead of using \pyinline{heapq}, using \pyinline{deque}.
\begin{python}
def findItinerary(self, tickets):
    G = defaultdict(deque)
    for s, e in tickets:
        G[s].append(e)
    
    for s, l in G.items():
        G[s] = deque(sorted(l))
    
    ret = deque()
    self.dfs(G, "JFK", ret)
    return list(ret)

def dfs(self, G, cur, ret):
    while G[cur]:
        # need to remove the edge after visting
        nbr = G[cur].popleft()
        self.dfs(G, nbr, ret)
    
    ret.appendleft(cur)
\end{python}

\subsection{Hamiltonian Path - Every Vertex Once} 

A Hamiltonian path is a path in a graph which visits every vertex exactly once ($\forall v \in V$). This problem is proved to be NP.

\section{Topological Sorting}
For a graph $G=\{V, E\}$, if $A \rightarrow B $, then $A$ is before $B$ in the ordered list.
\subsection{Algorithm}
\rih{Core clues}:
\begin{enumerate}
\item \textbf{DFS neighbors first}. For a given vertex $v$, if the neighbors of current node is  $\neg$visited, then dfs the neighbors
\item \textbf{Process current node}. After visiting all the neighbors, then visit the current node $v$ and push it to the result queue \pyinline{appendleft}.

\end{enumerate}
Notice:
\begin{enumerate}
\item Need to check ascending order or descending order.
\item Need to \textbf{detect cycle}; thus the dfs need to construct result queue and detect cycle simultaneously, by using two sets: $visited$ and $path$.
\end{enumerate}

\begin{python}
from collections import deque

def topological_sort(self, V):
  visited = set()
  ret = deque()

  for v in V.keys():
    if v not in visited:
      if not self.dfs_topo(V, v, visited, set(), ret):
        return []  # contains cycle

  return list(ret)

# return whether the current path is acyclic
def dfs_topo(self, V, v, visited, path, ret) -> bool:
  if v in path:
    return False

  path.add(v)
  for nbr in V[v]:
    if nbr not in visited:
      if not self.dfs_topo(V, nbr, visited, path, ret):
        return False

  path.remove(v)
  visited.add(v)
  ret.appendleft(v)
  return True
\end{python}
Alternatively, we can encode \pyinline{path} into \pyinline{visited}. 
\begin{python}
# encode the visited using 0, 1, 2
def dfs_topo(
  self, 
  G: Dict[int, List[int]], 
  u: int, 
  visited: Dict[int, int],
  ret: deque,
):
  """
  Topological sort
  G = defaultdict(list)
  visited = defaultdict(int) 
  # 0 not visited, 1 visiting, 2 visited

  pre-condition: u is not visited (0)
  """
  visited[u] = 1
  for nbr in G[u]:
    if visited[nbr] == 1:
      return False
    if visited[nbr] == 0:
      if not self.topo_dfs(G, nbr, visited, ret):
        return False

  visited[u] = 2
  ret.appendleft(u)  # visit larger first
  return True
\end{python}
The time complexity of topological sorting is $O(|E|+|V|)$ since it needs to goes to every edges and every vertices. 

\subsection{Applications}
\begin{enumerate}
\item Course scheduling problem with pre-requisite.
\end{enumerate}

\section{Union-Find}\label{section:unionFind}
\subsection{Simplified Union Find}
Simplified code with unbalanced size. Union-find and disjoint set are used interchangeably. Union-find emphasizes on algorithm while disjoint set emphasizes on data structure.

\rih{Core clues:}
\begin{enumerate}
\item \textbf{$\pi$ array}:an array to store each item's predecessor pi. The predecessor are lazily updated to its ancestor. 
\item When \pyinline{x == pi[x]}, then \pyinline{x} is the ancestor (i.e. root).
\item Otherwise, \pyinline{pi[x] = find(pi[x])}
\end{enumerate}
\begin{python}
class DisjointSet:
    def __init__(self):
        self.pi = {}

    def union(self, x, y):
        pi_x = self.find(x)
        pi_y = self.find(y)
        self.pi[pi_y] = pi_x

    def find(self, x):
        if x not in self.pi:
          self.pi[x] = x
          return x
        
        pi = self.pi[x]
        if x != pi:
          self.pi[x] = self.root(pi)
        return self.pi[x]
\end{python}

Note, for the final counting of component. Need to do a final find on all nodes to
update the ancestor.
\begin{python}
component_cnt = len(set(
    ds.find(x)
    for x in ds.pi.keys()
))
\end{python}

Improvements:
\begin{enumerate}
\item Weighting: size-baladnced tree
\item Path Compression.
\end{enumerate}
\subsection{Algorithm}
Weighted union-find with path compression.\\
\rih{Core clues:}
\begin{enumerate}
\item \textbf{$\pi$ array}: predecessor pi.
\item \textbf{Size-balanced}: merge the tree according to the size to maintain balance.
\item \textbf{Path compression}: Make the ptr in $\pi$ array to point to its root rather than its immediate parent.
\end{enumerate}
\begin{figure}[]
\centering
\subfloat{\includegraphics[width=\linewidth]{uf}}
\caption{Weighted quick-union traces}
\label{fig:union_find}
\end{figure}

\begin{python}
class UnionFind:  # or DisjointSet
  def __init__(self):
    self.pi = {}  # item -> pi
    self.sz = {}  # root -> size

  def __len__(self):
    """number of unions"""
    return len(self.sz)  # only root nodes have size

  def add(self, x):
    if x not in self.pi:
      self.pi[x] = x
      self.sz[x] = 1

  def find(self, x):
    """path compression"""
    pi = self.pi[x]
    if x != pi:
      self.pi[x] = self.find(pi)
    return self.pi[x]

  def unionize(self, x, y):
    pi1 = self.find(x)
    pi2 = self.find(y)

    if pi1 != pi2:
      if self.sz[pi1] > self.sz[pi2]:
        pi1, pi2 = pi2, pi1
        # size balancing
      self.pi[pi1] = pi2
      self.sz[pi2] += self.sz[pi1]
      del self.sz[pi1]

  def isunion(self, x, y):
    if x not in self.pi or y not in self.pi:
      return False
    return self.find(x) == self.find(y)
\end{python}

\subsection{Complexity}
$m$ union-find with $n$ objects: $O(n)+m O(\lg n)$

\section{Axis Projection}
Project the mat dimension from 2D to 1D, using \textit{orthogonal axis}.

\runinhead{Smallest bounding box.} Given the location $(x, y)$ of one of the 1's, return the area of the smallest bounding box that encloses 1's.
$$
\begin{bmatrix}
0& 0& 1& 0 \\
0& 1& 1& 0 \\
0& 1& 0& 0 \\
\end{bmatrix}
$$

\rih{Clues:}
\begin{enumerate}
\item Project the 1's onto x-axis, binary search for the left bound and right bound of the bounding box. 
\item We don't pre-project the axis beforehand, since it will take $O(mn)$ to collect the projected 1d array. Instead, we only project it during binary search when checking the mid item. Checking takes $O(m)$, searching takes $O(\log n)$. 
\item Do the same for y-axis.
\end{enumerate}

Time complexity: $O(m\log n + n \log m)$, where $O(m), O(n)$ is for projection complexity.
\section{MST}
Minimum spanning tree.
\subsection{Kruskal's algorithm}
\textbf{Core clues:}
\begin{enumerate}
\item Vertices $v \in V$ are divided into different sets
\item Extract min edges to unionize the sets
\item Terminates when $\forall v\in V$ are in the same set.
\end{enumerate}
\textbf{Code:}
\begin{python}[mathescape]
def kruskal(G):
  ret = []
  uf = UnionFind()
  for v in G.V:
    uf.add(v)

  G.E.sort()  # sort by weights
  for u, v in G.E:
    if not uf.isunion(u, v):
      A.append((u, v))
      uf.unionize(u, v)
\end{python}
Complexity: $O(|E|\log |E|)$.
