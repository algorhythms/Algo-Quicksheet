\chapter{Sort}


\section{Introduction}
List of general algorithms:
\begin{enumerate}
\item Selection sort: invariant
\begin{enumerate}
\item Elements to the left of $i$ (including $i$) are fixed and in ascending order (fixed and sorted).
\item No element to the right of $i$ is smaller than any entry to the left of $i$ ($A[i]  \leq\min(A[i+1:n])$.
\end{enumerate}
\item Insertion sort: invariant
\begin{enumerate}
\item Elements to the left of $i$ (including $i$) are in ascending order (sorted).
\item Elements to the right of $i$ have not yet been seen.
\end{enumerate}
\item Shell sort: h-sort using insertion sort.
\item Quick sort: invariant
\begin{enumerate}
\item $|A_p|..\leq..|..unseen..|..\geq..|$ maintain the 3 subarrays.
\end{enumerate}
\item Heap sort: compared to quick sort it is guaranteed $O(N \lg N)$, compared to merge sort it is $O(1)$ extra space. 
\end{enumerate}

\section{Algorithms}
\subsection{Quick Sort}
\subsubsection{Normal pivoting}
The key part of quick sort is pivoting:
\begin{python}
def pivot(self, A, i, j):
    """
    pivoting algorithm:
    p | closed set | open set |
    | closed set p | open set |
    """
    p = i
    closed = p
    for ptr in xrange(i, j):
        if A[ptr] < A[p]:
            closed += 1
            A[ptr], A[closed] = A[closed], A[ptr]

    A[closed], A[p] = A[p], A[closed]
    return closed
\end{python}

Notice that this implementation goes $O(N^2)$ for arrays with all duplicates.

\textbf{Problem with duplicate keys}: it is important to stop scan at duplicate
keys (counter-intuitive); otherwise quick sort will goes $O(N^2)$ for the
array with all duplicate items, because the algorithm will put all items
equal to the $A[p]$ on \textbf{a single side}. 

Example: quadratic time to sort random arrays of 0s and 1s.

\subsubsection{Stop-at-equal pivoting}
Alternative pivoting implementation with optimization for duplicated keys:
\begin{python}
def pivot_optimized(self, A, lo, hi):
    """
    Fix the pivot as the 1st element
    Scan from left to right and right to left simultaneously
    Avoid the case that the algo goes O(N^2) with duplicated keys
    """
    p = lo
    i = lo
    j = hi
    while True:
        while True:
            i += 1
            if i >= hi or A[i] >= A[lo]:
                break
        while True:
            j -= 1
            if j < lo or A[j] <= A[lo]:
                break

        if i >= j:
            break

        A[i], A[j] = A[j], A[i]

    A[lo], A[j] = A[j], A[lo]
    return j

\end{python}
\subsubsection{3-way pivoting}
3-way pivoting: pivot the array into 3 subarrays: 

$|..\leq..|..=..|..unseen..|..\geq..|$ 
\begin{python}
def pivot_3way(self, A, lo, hi):
    lt = lo-1  # pointing to end of array LT
    gt = hi  # pointing to the end of array GT (reversed)

    v = A[lo]
    i = lo  # scanning pointer
    while i < gt:
        if A[i] < v:
            lt += 1
            A[lt], A[i] = A[i], A[lt]
            i += 1
        elif A[i] > v:
            gt -= 1
            A[gt], A[i] = A[i], A[gt]
        else:
            i += 1

    return lt+1, gt
\end{python}

\section{Stability}
Definition: a stable sort preserves the \textbf{relative order of items with equal keys} (scenario: sorted by time then sorted by location). 

Algorithms:
\begin{enumerate}
\item Stable
\begin{enumerate}
\item Merge sort
\item Insertion sort
\end{enumerate} 
\item Unstable
\begin{enumerate}
\item Selection sort
\item Shell sort
\item Quick sort
\item Heap sort
\end{enumerate}
\end{enumerate}
\textbf{Long-distance swap} operation is the key to find the unstable case during sorting. 
\begin{figure}[hbtp]
\centering
\subfloat{\includegraphics[scale=.60]{stable_sort}}
\caption{Stale sort vs. unstable sort}
\label{fig:trie} 
\end{figure}

\section{Applications}
\begin{enumerate}
\item Sort
\item Partial quick sort (selection), k-th largest elements 
\item Binary search
\item Find duplicates 
\item Graham scan
\item Data compression
\end{enumerate}


\section{Reversion}
If $a_i > a_j$ but $i<j$, then this is considered as 1 reversion. 

\subsection{Calculation}
MergeSort to calculate the \#reverse-ordered paris. The only difference from a normal merge sort is that - when pushing the 2nd half of the array to the place, you calculates the reversion generated by the element $A_2[i_2]$ compared to $A_1[i_1:]$.

\begin{python}
def merge(A1, n1, A2, n2, A, n):
    i = i1 = i2 =0
    count = 0
    while i < n:
        if i1 == n1:
            for i2 in xrange(i2, n2):
                A[i] = A2[i2]
                i += 1
        elif i2 == n2:
            for i1 in xrange(i1, n1):
                A[i] = A1[i1]
                i += 1
        else:
            # use array diagram to illustrate
            if A1[i1] > A2[i2]:  # push the A2 to A
                A[i] = A2[i2]
                # number of reverse-ordered pairs
                count += n1 - i1
                i += 1
                i2 += 1
            else:
                A[i] = A1[i1]
                i += 1
                i1 += 1

    return count

def merge_sort(a):
    n = len(a)
    if n == 1:
        return 0
    n1 = n/2
    n2 = n - n1
    a1 = a[:n1]
    a2 = a[n1:]

    count1 = merge_sort(a1)
    count2 = merge_sort(a2)
    count = count1 + count2 + merge(a1, n1, a2, n2, a, n)

    return count
\end{python}



\section{Considerations}
\begin{enumerate}
\item Stable?
\item Distinct keys?
\item Need guaranteed performance?
\item Linked list or arrays?
\item Caching system? (reference to neighboring cells in the array? 
\item Usually randomly ordered array?
(or partially sorted?)\item Parallel?
\item Deterministic?
\item Multiple key types?
\end{enumerate}

$O(N\lg N)$ is the lower bound of comparison-based sorting; but for other
contexts, we may not need $O(N \lg N)$:
\begin{enumerate}
\item Partially-ordered arrays: insertion sort to achieve $O(N)$. \textbf{Number of inversions}: 1 inversion $=$ 1 pair of keys that are out
of order.
\item Duplicate keys
\item Digital properties of keys: radix sort to achieve $O(N)$.
\end{enumerate}

\section{Summary}
\begin{figure}[hbtp]
\centering
\subfloat{\includegraphics[scale=1.10]{sort_summary}}
\caption{Sort summary}
\label{fig:trie} 
\end{figure}
