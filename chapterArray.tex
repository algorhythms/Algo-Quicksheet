\chapter{Array}
\section{Two-pointer Algorithm}
\runinhead{Container With Most Water.} Given coordinate $(i, a_i)$, find two lines, which together with x-axis forms a container, such that the container contains the most water.
\begin{figure}[hbtp]
\centering
\subfloat{\includegraphics[width=0.8\linewidth]{Container-With-Most-Water.jpg}}
\caption{Container with Most Water}
\label{fig:Container-With-Most-Water}
\end{figure}

When calculate water area, the water height $h$ is constrained by $\min(h_{left}, h_{right})$. Core clues:
\begin{enumerate}
\item \textbf{Two pointers}: $start$, $back$ at two ends. Calculate the current area
\item \textbf{Move one}: Move the shorter (lower height) pointer. 
\end{enumerate}

Why does this greedy algorithm work? When searching further, we always search the best one.  When searching the most-water container with length $l$, we always reach the maximum area. When shrinking the water length by 1, we always reach the higher height. 

If we don't drop the lower side, we will always be constrained by the lower side. 
\section{Circular Array}
This section describes common patterns for solving problems with circular arrays.

Normally, we should solve the linear problem and circular problem very differently.

\subsection{Circular max sum}
Linear (acyclic) problem can be solved linear with dp algorithm for maximum subarray sum - Section \ref{dpSequence}. 

The circular sum should use dp. 

Problem description: Given an integer array containing both positive and negative, find a continuous rotate subarray where the sum of numbers is the largest. Return the index of the first number and the index of the last number. 
\runinhead{Core clues:}
\begin{enumerate}
\item \textbf{State definitions}: 

\rih{Max subarray sum from index 0}. Construct left max sum $L_i$ for max sum over the $[0..i]$ with subarray starting at 0 (\textit{forward} starting from the left side). 

\rih{Max subarray sum from index -1}. Construct right max sum $R_i$ for max sum over the indexes $[i+1..n -1]$, with subarray ending at -1 (\textit{backward} starting from the right side). 

Notice, for the two max sums, the index ends AT or BEFORE $i$.

\item \textbf{Transition functions:}
\begin{align*}
L_i = \max\Big(L_{i-1}, sum(A[:i])\Big) \\ 
R_i = \max\Big(R_{i+1}, sum(A[i:])\Big)
\end{align*}

\item \textbf{Global result}: 
$$maxa = \max(R_i+L_{i-1}, \forall i)$$
\end{enumerate}

\subsection{Non-adjacent cell}
Maximum sum of non-adjacent cells in an array $A$. (House robbery problem)

To solve circular non-adjacent array problem in linear way, we should consider 2 cases:
\begin{enumerate}
\item Consider the $A_i$
\item Not consider the $A_i$ 
\end{enumerate}
and solve them using linear maximum sum of non-adjacent cells separately - Section \ref{dpSequence}. 
\begin{align*}
F_{i} = \max\big(F_{i-1}, F_{i-2}+A_{i-1}\big)
\end{align*}
\subsection{Binary search}
Searching for an element in a circular sorted array. Half of the array is sorted while the other half is not.
\begin{enumerate}
\item If $A[0] < A[mid]$, then all values in the first half of the array are sorted.
\item If $A[mid] < A[-1]$, then all values in the second half of the array are sorted.
\item Then \textit{derive and decide} whether to got the \textbf{sorted half} or the \textbf{unsorted half}.
\end{enumerate}
\section{Voting Algorithm}
\subsection{Majority Number}
\subsubsection{$\frac{1}{2}$ of the Size}
Given an array of integers, the majority number is the number that occurs more than half of the size of the array. 

Algorithm: Majority Vote Algorithm. Maintain a counter to count how many times the majority number appear more than any other elements before index $i$ and after re-initialization. Re-initialization happens when the counter drops to 0. 

Proof: Find majority number $x$ in $A$. Mathematically, find $x$ in array $A$ with length $n$ s.t. $cnt_x > n -cnt_x$. 

Find a \textbf{pair} $(a_i, a_j)$ in $A$, if $a_i \neq a_j$, delete both from $A$. The counter  still
holds that: $C^{A'}_x > |A'|-C^{A'}_x$. Proof, since $a_i\neq a_j$, at most 1 of them
equals $x$, then $C^{A'}_x$ decrements at most by 1, $|A'|$ decrements by 2.

To find such pair $(a_i, a_j), a_i\neq a_j$, linear time one-pass algorithm. That's
why \textit{Moore's voting algorithm} is correct.

At any time in the execution, let $A'$ be the prefix of $A$ that has been processed,
if $counter>0$, then keep track the candidate $x$'s counter, the $x$
is the majority number of $A'$.  If $counter =0$, then for $A'$ we can pair the elements
s.t. are all pairs has distinct element. Thus, it does not hold that $cnt_x>n-cnt_x$;
thus $x\in A'$. $\blacksquare$
 
Re-check: This algorithm needs to re-check the current number being counted is indeed the majority number.    

\begin{python}
def majorityElement(self, nums):
    """
    Algorithm:
    O(n lgn) sort and take the middle one
    O(n) Moore's Voting Algorithm
    """
    mjr = nums[0]
    cnt = 0
    for i, v in enumerate(nums):
        if mjr == v:
            cnt += 1
        else:
            cnt -= 1

        if cnt < 0:
            mjr = v
            cnt = 1

    return mjr

\end{python}
\subsubsection{$\frac{1}{3}$ of the Size}
Given an array of integers, the majority number is the number that occurs more than $\frac{1}{3}$ of the size of the array. This question can be generalized to be solved by $\frac{1}{k}$ case. 

\subsubsection{$\frac{1}{k}$ of the Size}
Given an array of integers and a number k, the majority number is the number that occurs more than $\frac{1}{k}$ of the size of the array. In this case, we need to generalize the solution to $\frac{1}{2}$ majority number problem.

\begin{python}

def majorityNumber(self, A, k):
    """
    Since majority elements appears more 
    than ceil(n/k) times, there are at 
    most k-1 majority number
    """
    cnt = defaultdict(int)
    for e in A:
        if e in cnt:
            cnt[e] += 1
        else:
            if len(cnt) < k-1:
                cnt[e] += 1
            else:
                for key in cnt.keys():
                    cnt[key] -= 1
                    if cnt[key] == 0:
                        del cnt[key]
    
    
    # filter, double-check
    for key in cnt.keys():
        if len(filter(lambda x: x == key, A)) 
            > len(A)/k:
            return key

    raise Exception
\end{python}


\section{Two Pointers}
\subsection{Interleaving}
\runinhead{Interleaving positive and negative numbers.} Given an array with positive and negative integers. Re-range it to interleaving with positive and negative integers.
\begin{lstlisting}
Input:
[-33, -19, 30, 26, 21, -9]
Output:
[-33, 30, -19, 26, -9, 21]
\end{lstlisting}
Core clues:
\begin{enumerate}
\item How to do it in O(N).
\item What (positive or negative) is expected for the current position.
\item Where is the next positive and negative element.
\end{enumerate}
\begin{python}
def rerange(self, A):
    n = len(A)
    pos_cnt = len(filter(lambda x: x > 0, A))
    pos_expt = True if pos_cnt*2 > n else False

    neg = 0  # next negative
    pos = 0  # next positive
    for i in range(n):
        # search for the next 
        while neg < n and A[neg] > 0: neg += 1
        while pos < n and A[pos] < 0: pos += 1
        
        if pos_expt:
            A[i], A[pos] = A[pos], A[i]
        else:
            A[i], A[neg] = A[neg], A[i]

        if i == neg: neg += 1
        if i == pos: pos += 1

        pos_expt = not pos_expt
\end{python}

\section{Index Remapping}
\subsection{Introduction}
\runinhead{Virtual Index.} Similar to physical and virtual machine, the underlying indexing $i$ for an array $A$ is the physical index, while we can create virtual indexing $i'$ for the same array $A$ to map $A_{i'}$ to a physical entry $A_{i}$.
\subsection{Example}
\runinhead{Interleaving indexes} Given an array $A$ of length $n$, we want to mapping the virtual indexes to physical indexes such that $A_0$ maps to $A_1$, $A_1$ maps to $A_3$,..., $A_{\lfloor n/2\rfloor}$ maps to $A_0$, as followed: 
\begin{figure}[hbtp]
\centering
\subfloat{\includegraphics[width=\linewidth]{virtual_indexes.png}}
\caption{Virtual Indices Remapping}
\label{fig:virtual_indexes}
\end{figure}
\begin{lstlisting}
0 -> 1
1 -> 3
2 -> 5
...
n/2-1 -> n-1 or n-2

n/2 -> 0
n/2+1 -> 2
...
n -> n-2 or n-1
\end{lstlisting}
If $n$ is even, 

$$
(2*i+1)\%(n+1)
$$

If $n$ is odd,

$$
(2*i+1)\%(n)
$$

Thus, by combining two cases, we create the mapping relationship: 
\begin{python}
def idx(i):
    return (2*i+1) % (n|1)
\end{python}

