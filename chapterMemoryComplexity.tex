\chapter{Memory Complexity}

\section{Introduction}
The memory usage is based on Java.

\begin{longtable}{ll}
\hline\noalign{\smallskip}
\textbf{Type} & \textbf{Bytes} \\
\noalign{\smallskip}\hline\noalign{\smallskip}

boolean & 1 \\
byte & 1 \\
char & 2 \\
int & 4 \\
float & 4 \\
long & 8 \\
double & 8\\

\noalign{\smallskip}\hline\noalign{\smallskip}
\caption {for primitive types}
\end{longtable}

\begin{longtable}{ll}
\hline\noalign{\smallskip}
\textbf{Type} & \textbf{Bytes} \\
\noalign{\smallskip}\hline\noalign{\smallskip}

char[] & 2N+24 \\
int[] & 4N+24 \\
double[] & 8N+24 \\

\noalign{\smallskip}\hline\noalign{\smallskip}
\caption {for one-dimensional arrays}
\end{longtable}

\begin{longtable}{ll}
\hline\noalign{\smallskip}
\textbf{Type} & \textbf{Bytes} \\
\noalign{\smallskip}\hline\noalign{\smallskip}

char[][] & 2MN \\
int[][] & 4MN \\
double[][] & 8MN \\

\noalign{\smallskip}\hline\noalign{\smallskip}
\caption {for two-dimensional arrays}
\end{longtable}

\begin{longtable}{ll}
\hline\noalign{\smallskip}
\textbf{Type} & \textbf{Bytes} \\
\noalign{\smallskip}\hline\noalign{\smallskip}

Object overhead & 16 \\
Reference & 8 \\
Padding & 8x \\

\noalign{\smallskip}\hline\noalign{\smallskip}
\caption {for objects}
\end{longtable}
