\chapter{Bit Manipulation}
\section{Concepts}
\subsection{Basics}
\begin{enumerate}
\item Bit value: bit0, bit1. 
\item BitSet/Bits
\item Bit position (bit interchangeably)
\item 32-bit signed range: $[-2^{31}, 2^{31}-1]$. $0$ is like positive number without complement. 
\end{enumerate}
\subsection{Operations}
\runinhead{Mask.} 
\begin{enumerate}
\item Masking to 1: to mask a single bit position, $bit\OR 1$
\item Masking to 0: to mask a single bit position, $bit\AND 0$
\item Querying a bit position value: to query a single bit position, $bit\AND 0010$
\item Toggling bit values: to toggle a single bit position, $bit\XOR 1$
\end{enumerate}}
This can be extended to do masking operations on multiple bits. 

\runinhead{Check 2's power}
$$x\AND(x-1)$$

\runinhead{Rightmost bit set.} To get the rightmost bit, with the help of 2's complement:
\begin{enumerate}
\item Left extended with 1's:
$$x \XOR (-x)$$
\item Left extended with 0's:
$$x \AND (-x)$$
\end{enumerate}

\runinhead{Negation and index} We can use tilde notation for the index accessing a string or an array
\begin{lstlisting}
i  ~i  
0  -1
1  -2
2  -3
3  -4 
4  -5 
5  -6
\end{lstlisting}
$$
\NOT i = -i+1
$$
To determine whether a string is palindrome:
\begin{python}
def is_palindrome(s):
  return all(s[i] == s[~i] for i in xrange(len(s)/2)) 
\end{python}
\section{Single Number}
\subsection{Three-time appearance} 
Given an array of integers, every element appears three times except for one. Find that single one.

\rih{Using list.} Consider 4-bit numbers:
\begin{eqnarray*}
&& 0000 \\
&& 0001 \\
&& 0010 \\
&& ... \\
&& 1111
\end{eqnarray*}

Add (not $\&$) the bit values \textbf{vertically}, then result would be $abcd$ where $a, b, c, d$ can be any number, not just binary. $a, b, c, d$ can be divided by 3 if the all element appears three times. Until here, you can use a list to hold $a, b, c, d$. By mod 3, the single one that does not appear 3 times is found. 

To generalize to 32-bit \pythoninline{int}, use a list of length 32.

\rih{Using bits.}
To further optimize the space, use bits (bit set) instead of list. 
\begin{itemize}
\item Since all except one appears 3 times, we are only interested in $0, 1, 2$ (mod 3) count of bit1 appearances in a bit position.
\item We create 3 bit sets to represent $0, 1, 2$ appearances of all positions of bits.
\item For a bit, there is one and only one bit set containing bit1 in that bit position.
\item Transition among the 3 bit sets for every number:
$$
bitSet^{(i)} = (bitSet^{(i-1)}\AND num)\OR(bitSet^{(i)}\AND \NOT num)
$$
\end{itemize}

For $i$ appearances, the first part is the bit set \textbf{transited from} $(i-1)$ appearances, and the second part is the bit set \textbf{transited out} from itself.

Consider each single bit separately. For the $j$-th bit in $num$, if $num_j=1$, the first part indicates $bitSet^{(i-1)}$ will transit in (since transition); the 2nd part is always 0 (since transition out or initially 0). If $num_j=0$, the 1st part is always 0 (since no transition); the 2nd part indicates $bitSet^{(i)}$ will remain the same (since no transition). 



\subsection{Two Numbers} 
Given an array of numbers nums, in which exactly two elements appear only once and all the other elements appear exactly twice. Find the two elements that appear only once.

\begin{itemize}
\item Easily get: $x = a \XOR b$.
\item $a \neq b$; thus there are at least one 1-bit in $x$ is different.  
\item Take an arbitrary 1 bit set in $x$, and such bit set can classify the elements in the array into two separate groups.
\end{itemize}

\section{Bitwise operators}
\runinhead{Comparison.} Write a method which finds the maximum of two numbers $a, b$. You should not use if- else or any other comparison operator
\\
Clues:
\begin{enumerate}
\item check the sign bit $s$ of $a-b$.
\item return $a-s*(a-b)$
\end{enumerate}
Codes:
\begin{java}
int getMax(int a, int b) { 
    int c = a - b;
    int k = (c >> 31) & 0x1; 
    int max = a - k * c; 
    return max;
}

\end{java}
If consider overflow, it raises another level of difficulty. 
